
% Default to the notebook output style

    


% Inherit from the specified cell style.




    
\documentclass[11pt]{article}

    
    
    \usepackage[T1]{fontenc}
    % Nicer default font (+ math font) than Computer Modern for most use cases
    \usepackage{mathpazo}

    % Basic figure setup, for now with no caption control since it's done
    % automatically by Pandoc (which extracts ![](path) syntax from Markdown).
    \usepackage{graphicx}
    % We will generate all images so they have a width \maxwidth. This means
    % that they will get their normal width if they fit onto the page, but
    % are scaled down if they would overflow the margins.
    \makeatletter
    \def\maxwidth{\ifdim\Gin@nat@width>\linewidth\linewidth
    \else\Gin@nat@width\fi}
    \makeatother
    \let\Oldincludegraphics\includegraphics
    % Set max figure width to be 80% of text width, for now hardcoded.
    \renewcommand{\includegraphics}[1]{\Oldincludegraphics[width=.8\maxwidth]{#1}}
    % Ensure that by default, figures have no caption (until we provide a
    % proper Figure object with a Caption API and a way to capture that
    % in the conversion process - todo).
    \usepackage{caption}
    \DeclareCaptionLabelFormat{nolabel}{}
    \captionsetup{labelformat=nolabel}

    \usepackage{adjustbox} % Used to constrain images to a maximum size 
    \usepackage{xcolor} % Allow colors to be defined
    \usepackage{enumerate} % Needed for markdown enumerations to work
    \usepackage{geometry} % Used to adjust the document margins
    \usepackage{amsmath} % Equations
    \usepackage{amssymb} % Equations
    \usepackage{textcomp} % defines textquotesingle
    % Hack from http://tex.stackexchange.com/a/47451/13684:
    \AtBeginDocument{%
        \def\PYZsq{\textquotesingle}% Upright quotes in Pygmentized code
    }
    \usepackage{upquote} % Upright quotes for verbatim code
    \usepackage{eurosym} % defines \euro
    \usepackage[mathletters]{ucs} % Extended unicode (utf-8) support
    \usepackage[utf8x]{inputenc} % Allow utf-8 characters in the tex document
    \usepackage{fancyvrb} % verbatim replacement that allows latex
    \usepackage{grffile} % extends the file name processing of package graphics 
                         % to support a larger range 
    % The hyperref package gives us a pdf with properly built
    % internal navigation ('pdf bookmarks' for the table of contents,
    % internal cross-reference links, web links for URLs, etc.)
    \usepackage{hyperref}
    \usepackage{longtable} % longtable support required by pandoc >1.10
    \usepackage{booktabs}  % table support for pandoc > 1.12.2
    \usepackage[inline]{enumitem} % IRkernel/repr support (it uses the enumerate* environment)
    \usepackage[normalem]{ulem} % ulem is needed to support strikethroughs (\sout)
                                % normalem makes italics be italics, not underlines
    

    
    
    % Colors for the hyperref package
    \definecolor{urlcolor}{rgb}{0,.145,.698}
    \definecolor{linkcolor}{rgb}{.71,0.21,0.01}
    \definecolor{citecolor}{rgb}{.12,.54,.11}

    % ANSI colors
    \definecolor{ansi-black}{HTML}{3E424D}
    \definecolor{ansi-black-intense}{HTML}{282C36}
    \definecolor{ansi-red}{HTML}{E75C58}
    \definecolor{ansi-red-intense}{HTML}{B22B31}
    \definecolor{ansi-green}{HTML}{00A250}
    \definecolor{ansi-green-intense}{HTML}{007427}
    \definecolor{ansi-yellow}{HTML}{DDB62B}
    \definecolor{ansi-yellow-intense}{HTML}{B27D12}
    \definecolor{ansi-blue}{HTML}{208FFB}
    \definecolor{ansi-blue-intense}{HTML}{0065CA}
    \definecolor{ansi-magenta}{HTML}{D160C4}
    \definecolor{ansi-magenta-intense}{HTML}{A03196}
    \definecolor{ansi-cyan}{HTML}{60C6C8}
    \definecolor{ansi-cyan-intense}{HTML}{258F8F}
    \definecolor{ansi-white}{HTML}{C5C1B4}
    \definecolor{ansi-white-intense}{HTML}{A1A6B2}

    % commands and environments needed by pandoc snippets
    % extracted from the output of `pandoc -s`
    \providecommand{\tightlist}{%
      \setlength{\itemsep}{0pt}\setlength{\parskip}{0pt}}
    \DefineVerbatimEnvironment{Highlighting}{Verbatim}{commandchars=\\\{\}}
    % Add ',fontsize=\small' for more characters per line
    \newenvironment{Shaded}{}{}
    \newcommand{\KeywordTok}[1]{\textcolor[rgb]{0.00,0.44,0.13}{\textbf{{#1}}}}
    \newcommand{\DataTypeTok}[1]{\textcolor[rgb]{0.56,0.13,0.00}{{#1}}}
    \newcommand{\DecValTok}[1]{\textcolor[rgb]{0.25,0.63,0.44}{{#1}}}
    \newcommand{\BaseNTok}[1]{\textcolor[rgb]{0.25,0.63,0.44}{{#1}}}
    \newcommand{\FloatTok}[1]{\textcolor[rgb]{0.25,0.63,0.44}{{#1}}}
    \newcommand{\CharTok}[1]{\textcolor[rgb]{0.25,0.44,0.63}{{#1}}}
    \newcommand{\StringTok}[1]{\textcolor[rgb]{0.25,0.44,0.63}{{#1}}}
    \newcommand{\CommentTok}[1]{\textcolor[rgb]{0.38,0.63,0.69}{\textit{{#1}}}}
    \newcommand{\OtherTok}[1]{\textcolor[rgb]{0.00,0.44,0.13}{{#1}}}
    \newcommand{\AlertTok}[1]{\textcolor[rgb]{1.00,0.00,0.00}{\textbf{{#1}}}}
    \newcommand{\FunctionTok}[1]{\textcolor[rgb]{0.02,0.16,0.49}{{#1}}}
    \newcommand{\RegionMarkerTok}[1]{{#1}}
    \newcommand{\ErrorTok}[1]{\textcolor[rgb]{1.00,0.00,0.00}{\textbf{{#1}}}}
    \newcommand{\NormalTok}[1]{{#1}}
    
    % Additional commands for more recent versions of Pandoc
    \newcommand{\ConstantTok}[1]{\textcolor[rgb]{0.53,0.00,0.00}{{#1}}}
    \newcommand{\SpecialCharTok}[1]{\textcolor[rgb]{0.25,0.44,0.63}{{#1}}}
    \newcommand{\VerbatimStringTok}[1]{\textcolor[rgb]{0.25,0.44,0.63}{{#1}}}
    \newcommand{\SpecialStringTok}[1]{\textcolor[rgb]{0.73,0.40,0.53}{{#1}}}
    \newcommand{\ImportTok}[1]{{#1}}
    \newcommand{\DocumentationTok}[1]{\textcolor[rgb]{0.73,0.13,0.13}{\textit{{#1}}}}
    \newcommand{\AnnotationTok}[1]{\textcolor[rgb]{0.38,0.63,0.69}{\textbf{\textit{{#1}}}}}
    \newcommand{\CommentVarTok}[1]{\textcolor[rgb]{0.38,0.63,0.69}{\textbf{\textit{{#1}}}}}
    \newcommand{\VariableTok}[1]{\textcolor[rgb]{0.10,0.09,0.49}{{#1}}}
    \newcommand{\ControlFlowTok}[1]{\textcolor[rgb]{0.00,0.44,0.13}{\textbf{{#1}}}}
    \newcommand{\OperatorTok}[1]{\textcolor[rgb]{0.40,0.40,0.40}{{#1}}}
    \newcommand{\BuiltInTok}[1]{{#1}}
    \newcommand{\ExtensionTok}[1]{{#1}}
    \newcommand{\PreprocessorTok}[1]{\textcolor[rgb]{0.74,0.48,0.00}{{#1}}}
    \newcommand{\AttributeTok}[1]{\textcolor[rgb]{0.49,0.56,0.16}{{#1}}}
    \newcommand{\InformationTok}[1]{\textcolor[rgb]{0.38,0.63,0.69}{\textbf{\textit{{#1}}}}}
    \newcommand{\WarningTok}[1]{\textcolor[rgb]{0.38,0.63,0.69}{\textbf{\textit{{#1}}}}}
    
    
    % Define a nice break command that doesn't care if a line doesn't already
    % exist.
    \def\br{\hspace*{\fill} \\* }
    % Math Jax compatability definitions
    \def\gt{>}
    \def\lt{<}
    % Document parameters
    \title{cxx-oop-partII}
    
    
    

    % Pygments definitions
    
\makeatletter
\def\PY@reset{\let\PY@it=\relax \let\PY@bf=\relax%
    \let\PY@ul=\relax \let\PY@tc=\relax%
    \let\PY@bc=\relax \let\PY@ff=\relax}
\def\PY@tok#1{\csname PY@tok@#1\endcsname}
\def\PY@toks#1+{\ifx\relax#1\empty\else%
    \PY@tok{#1}\expandafter\PY@toks\fi}
\def\PY@do#1{\PY@bc{\PY@tc{\PY@ul{%
    \PY@it{\PY@bf{\PY@ff{#1}}}}}}}
\def\PY#1#2{\PY@reset\PY@toks#1+\relax+\PY@do{#2}}

\expandafter\def\csname PY@tok@w\endcsname{\def\PY@tc##1{\textcolor[rgb]{0.73,0.73,0.73}{##1}}}
\expandafter\def\csname PY@tok@c\endcsname{\let\PY@it=\textit\def\PY@tc##1{\textcolor[rgb]{0.25,0.50,0.50}{##1}}}
\expandafter\def\csname PY@tok@cp\endcsname{\def\PY@tc##1{\textcolor[rgb]{0.74,0.48,0.00}{##1}}}
\expandafter\def\csname PY@tok@k\endcsname{\let\PY@bf=\textbf\def\PY@tc##1{\textcolor[rgb]{0.00,0.50,0.00}{##1}}}
\expandafter\def\csname PY@tok@kp\endcsname{\def\PY@tc##1{\textcolor[rgb]{0.00,0.50,0.00}{##1}}}
\expandafter\def\csname PY@tok@kt\endcsname{\def\PY@tc##1{\textcolor[rgb]{0.69,0.00,0.25}{##1}}}
\expandafter\def\csname PY@tok@o\endcsname{\def\PY@tc##1{\textcolor[rgb]{0.40,0.40,0.40}{##1}}}
\expandafter\def\csname PY@tok@ow\endcsname{\let\PY@bf=\textbf\def\PY@tc##1{\textcolor[rgb]{0.67,0.13,1.00}{##1}}}
\expandafter\def\csname PY@tok@nb\endcsname{\def\PY@tc##1{\textcolor[rgb]{0.00,0.50,0.00}{##1}}}
\expandafter\def\csname PY@tok@nf\endcsname{\def\PY@tc##1{\textcolor[rgb]{0.00,0.00,1.00}{##1}}}
\expandafter\def\csname PY@tok@nc\endcsname{\let\PY@bf=\textbf\def\PY@tc##1{\textcolor[rgb]{0.00,0.00,1.00}{##1}}}
\expandafter\def\csname PY@tok@nn\endcsname{\let\PY@bf=\textbf\def\PY@tc##1{\textcolor[rgb]{0.00,0.00,1.00}{##1}}}
\expandafter\def\csname PY@tok@ne\endcsname{\let\PY@bf=\textbf\def\PY@tc##1{\textcolor[rgb]{0.82,0.25,0.23}{##1}}}
\expandafter\def\csname PY@tok@nv\endcsname{\def\PY@tc##1{\textcolor[rgb]{0.10,0.09,0.49}{##1}}}
\expandafter\def\csname PY@tok@no\endcsname{\def\PY@tc##1{\textcolor[rgb]{0.53,0.00,0.00}{##1}}}
\expandafter\def\csname PY@tok@nl\endcsname{\def\PY@tc##1{\textcolor[rgb]{0.63,0.63,0.00}{##1}}}
\expandafter\def\csname PY@tok@ni\endcsname{\let\PY@bf=\textbf\def\PY@tc##1{\textcolor[rgb]{0.60,0.60,0.60}{##1}}}
\expandafter\def\csname PY@tok@na\endcsname{\def\PY@tc##1{\textcolor[rgb]{0.49,0.56,0.16}{##1}}}
\expandafter\def\csname PY@tok@nt\endcsname{\let\PY@bf=\textbf\def\PY@tc##1{\textcolor[rgb]{0.00,0.50,0.00}{##1}}}
\expandafter\def\csname PY@tok@nd\endcsname{\def\PY@tc##1{\textcolor[rgb]{0.67,0.13,1.00}{##1}}}
\expandafter\def\csname PY@tok@s\endcsname{\def\PY@tc##1{\textcolor[rgb]{0.73,0.13,0.13}{##1}}}
\expandafter\def\csname PY@tok@sd\endcsname{\let\PY@it=\textit\def\PY@tc##1{\textcolor[rgb]{0.73,0.13,0.13}{##1}}}
\expandafter\def\csname PY@tok@si\endcsname{\let\PY@bf=\textbf\def\PY@tc##1{\textcolor[rgb]{0.73,0.40,0.53}{##1}}}
\expandafter\def\csname PY@tok@se\endcsname{\let\PY@bf=\textbf\def\PY@tc##1{\textcolor[rgb]{0.73,0.40,0.13}{##1}}}
\expandafter\def\csname PY@tok@sr\endcsname{\def\PY@tc##1{\textcolor[rgb]{0.73,0.40,0.53}{##1}}}
\expandafter\def\csname PY@tok@ss\endcsname{\def\PY@tc##1{\textcolor[rgb]{0.10,0.09,0.49}{##1}}}
\expandafter\def\csname PY@tok@sx\endcsname{\def\PY@tc##1{\textcolor[rgb]{0.00,0.50,0.00}{##1}}}
\expandafter\def\csname PY@tok@m\endcsname{\def\PY@tc##1{\textcolor[rgb]{0.40,0.40,0.40}{##1}}}
\expandafter\def\csname PY@tok@gh\endcsname{\let\PY@bf=\textbf\def\PY@tc##1{\textcolor[rgb]{0.00,0.00,0.50}{##1}}}
\expandafter\def\csname PY@tok@gu\endcsname{\let\PY@bf=\textbf\def\PY@tc##1{\textcolor[rgb]{0.50,0.00,0.50}{##1}}}
\expandafter\def\csname PY@tok@gd\endcsname{\def\PY@tc##1{\textcolor[rgb]{0.63,0.00,0.00}{##1}}}
\expandafter\def\csname PY@tok@gi\endcsname{\def\PY@tc##1{\textcolor[rgb]{0.00,0.63,0.00}{##1}}}
\expandafter\def\csname PY@tok@gr\endcsname{\def\PY@tc##1{\textcolor[rgb]{1.00,0.00,0.00}{##1}}}
\expandafter\def\csname PY@tok@ge\endcsname{\let\PY@it=\textit}
\expandafter\def\csname PY@tok@gs\endcsname{\let\PY@bf=\textbf}
\expandafter\def\csname PY@tok@gp\endcsname{\let\PY@bf=\textbf\def\PY@tc##1{\textcolor[rgb]{0.00,0.00,0.50}{##1}}}
\expandafter\def\csname PY@tok@go\endcsname{\def\PY@tc##1{\textcolor[rgb]{0.53,0.53,0.53}{##1}}}
\expandafter\def\csname PY@tok@gt\endcsname{\def\PY@tc##1{\textcolor[rgb]{0.00,0.27,0.87}{##1}}}
\expandafter\def\csname PY@tok@err\endcsname{\def\PY@bc##1{\setlength{\fboxsep}{0pt}\fcolorbox[rgb]{1.00,0.00,0.00}{1,1,1}{\strut ##1}}}
\expandafter\def\csname PY@tok@kc\endcsname{\let\PY@bf=\textbf\def\PY@tc##1{\textcolor[rgb]{0.00,0.50,0.00}{##1}}}
\expandafter\def\csname PY@tok@kd\endcsname{\let\PY@bf=\textbf\def\PY@tc##1{\textcolor[rgb]{0.00,0.50,0.00}{##1}}}
\expandafter\def\csname PY@tok@kn\endcsname{\let\PY@bf=\textbf\def\PY@tc##1{\textcolor[rgb]{0.00,0.50,0.00}{##1}}}
\expandafter\def\csname PY@tok@kr\endcsname{\let\PY@bf=\textbf\def\PY@tc##1{\textcolor[rgb]{0.00,0.50,0.00}{##1}}}
\expandafter\def\csname PY@tok@bp\endcsname{\def\PY@tc##1{\textcolor[rgb]{0.00,0.50,0.00}{##1}}}
\expandafter\def\csname PY@tok@fm\endcsname{\def\PY@tc##1{\textcolor[rgb]{0.00,0.00,1.00}{##1}}}
\expandafter\def\csname PY@tok@vc\endcsname{\def\PY@tc##1{\textcolor[rgb]{0.10,0.09,0.49}{##1}}}
\expandafter\def\csname PY@tok@vg\endcsname{\def\PY@tc##1{\textcolor[rgb]{0.10,0.09,0.49}{##1}}}
\expandafter\def\csname PY@tok@vi\endcsname{\def\PY@tc##1{\textcolor[rgb]{0.10,0.09,0.49}{##1}}}
\expandafter\def\csname PY@tok@vm\endcsname{\def\PY@tc##1{\textcolor[rgb]{0.10,0.09,0.49}{##1}}}
\expandafter\def\csname PY@tok@sa\endcsname{\def\PY@tc##1{\textcolor[rgb]{0.73,0.13,0.13}{##1}}}
\expandafter\def\csname PY@tok@sb\endcsname{\def\PY@tc##1{\textcolor[rgb]{0.73,0.13,0.13}{##1}}}
\expandafter\def\csname PY@tok@sc\endcsname{\def\PY@tc##1{\textcolor[rgb]{0.73,0.13,0.13}{##1}}}
\expandafter\def\csname PY@tok@dl\endcsname{\def\PY@tc##1{\textcolor[rgb]{0.73,0.13,0.13}{##1}}}
\expandafter\def\csname PY@tok@s2\endcsname{\def\PY@tc##1{\textcolor[rgb]{0.73,0.13,0.13}{##1}}}
\expandafter\def\csname PY@tok@sh\endcsname{\def\PY@tc##1{\textcolor[rgb]{0.73,0.13,0.13}{##1}}}
\expandafter\def\csname PY@tok@s1\endcsname{\def\PY@tc##1{\textcolor[rgb]{0.73,0.13,0.13}{##1}}}
\expandafter\def\csname PY@tok@mb\endcsname{\def\PY@tc##1{\textcolor[rgb]{0.40,0.40,0.40}{##1}}}
\expandafter\def\csname PY@tok@mf\endcsname{\def\PY@tc##1{\textcolor[rgb]{0.40,0.40,0.40}{##1}}}
\expandafter\def\csname PY@tok@mh\endcsname{\def\PY@tc##1{\textcolor[rgb]{0.40,0.40,0.40}{##1}}}
\expandafter\def\csname PY@tok@mi\endcsname{\def\PY@tc##1{\textcolor[rgb]{0.40,0.40,0.40}{##1}}}
\expandafter\def\csname PY@tok@il\endcsname{\def\PY@tc##1{\textcolor[rgb]{0.40,0.40,0.40}{##1}}}
\expandafter\def\csname PY@tok@mo\endcsname{\def\PY@tc##1{\textcolor[rgb]{0.40,0.40,0.40}{##1}}}
\expandafter\def\csname PY@tok@ch\endcsname{\let\PY@it=\textit\def\PY@tc##1{\textcolor[rgb]{0.25,0.50,0.50}{##1}}}
\expandafter\def\csname PY@tok@cm\endcsname{\let\PY@it=\textit\def\PY@tc##1{\textcolor[rgb]{0.25,0.50,0.50}{##1}}}
\expandafter\def\csname PY@tok@cpf\endcsname{\let\PY@it=\textit\def\PY@tc##1{\textcolor[rgb]{0.25,0.50,0.50}{##1}}}
\expandafter\def\csname PY@tok@c1\endcsname{\let\PY@it=\textit\def\PY@tc##1{\textcolor[rgb]{0.25,0.50,0.50}{##1}}}
\expandafter\def\csname PY@tok@cs\endcsname{\let\PY@it=\textit\def\PY@tc##1{\textcolor[rgb]{0.25,0.50,0.50}{##1}}}

\def\PYZbs{\char`\\}
\def\PYZus{\char`\_}
\def\PYZob{\char`\{}
\def\PYZcb{\char`\}}
\def\PYZca{\char`\^}
\def\PYZam{\char`\&}
\def\PYZlt{\char`\<}
\def\PYZgt{\char`\>}
\def\PYZsh{\char`\#}
\def\PYZpc{\char`\%}
\def\PYZdl{\char`\$}
\def\PYZhy{\char`\-}
\def\PYZsq{\char`\'}
\def\PYZdq{\char`\"}
\def\PYZti{\char`\~}
% for compatibility with earlier versions
\def\PYZat{@}
\def\PYZlb{[}
\def\PYZrb{]}
\makeatother


    % Exact colors from NB
    \definecolor{incolor}{rgb}{0.0, 0.0, 0.5}
    \definecolor{outcolor}{rgb}{0.545, 0.0, 0.0}



    
    % Prevent overflowing lines due to hard-to-break entities
    \sloppy 
    % Setup hyperref package
    \hypersetup{
      breaklinks=true,  % so long urls are correctly broken across lines
      colorlinks=true,
      urlcolor=urlcolor,
      linkcolor=linkcolor,
      citecolor=citecolor,
      }
    % Slightly bigger margins than the latex defaults
    
    \geometry{verbose,tmargin=1in,bmargin=1in,lmargin=1in,rmargin=1in}
    
    

    \begin{document}
    
    
    \maketitle
    
    

    
    \hypertarget{oop-part-ii-c}{%
\section{OOP Part II: C++}\label{oop-part-ii-c}}

\hypertarget{topics}{%
\subsection{Topics}\label{topics}}

\begin{enumerate}
\def\labelenumi{\arabic{enumi}.}
\tightlist
\item
  Recap: Encapsulation
\item
  Recap: Abstraction
\item
  Inheritance

  \begin{itemize}
  \tightlist
  \item
    virtual methods
  \end{itemize}
\item
  Polymorphism
\item
  Abstract classes

  \begin{itemize}
  \tightlist
  \item
    pure virtual methods
  \end{itemize}
\item
  Composition
\end{enumerate}

\hypertarget{recap-encapsulation}{%
\subsection{1. Recap: Encapsulation}\label{recap-encapsulation}}

\begin{itemize}
\tightlist
\item
  Encapsulation hides the details of an object's state from unauthorized
  parties.
\item
  Only the object's instance and its methods have access to these
  values.
\item
  Achieved in C++ by maintaining private member variables
\end{itemize}

    \begin{Verbatim}[commandchars=\\\{\}]
{\color{incolor}In [{\color{incolor} }]:} \PY{c+cp}{\PYZsh{}}\PY{c+cp}{include} \PY{c+cpf}{\PYZlt{}iostream\PYZgt{}}
\end{Verbatim}


    \begin{Verbatim}[commandchars=\\\{\}]
{\color{incolor}In [{\color{incolor} }]:} \PY{k}{class} \PY{n+nc}{Student\PYZus{}Encapsulation\PYZus{}Example} \PY{p}{\PYZob{}}
            \PY{c+c1}{// By default, these variables are private}
            \PY{k+kt}{int} \PY{n}{id\PYZus{}}\PY{p}{;}
            \PY{k+kt}{double} \PY{n}{gpa\PYZus{}}\PY{p}{;}
            \PY{n}{std}\PY{o}{:}\PY{o}{:}\PY{n}{vector}\PY{o}{\PYZlt{}}\PY{n}{std}\PY{o}{:}\PY{o}{:}\PY{n}{string}\PY{o}{\PYZgt{}} \PY{n}{courses\PYZus{}}\PY{p}{;}
          
          \PY{k}{public}\PY{o}{:}
            \PY{n}{Student\PYZus{}Encapsulation\PYZus{}Example}\PY{p}{(}\PY{k+kt}{int} \PY{n}{id}\PY{p}{)}\PY{p}{\PYZob{}}
                \PY{k}{this}\PY{o}{\PYZhy{}}\PY{o}{\PYZgt{}}\PY{n}{id\PYZus{}} \PY{o}{=} \PY{n}{id}\PY{p}{;}
            \PY{p}{\PYZcb{}}\PY{p}{;}
        \PY{p}{\PYZcb{}}
\end{Verbatim}


    \begin{Verbatim}[commandchars=\\\{\}]
{\color{incolor}In [{\color{incolor} }]:} \PY{n}{Student\PYZus{}Encapsulation\PYZus{}Example} \PY{n}{s} \PY{o}{=} \PY{n}{Student\PYZus{}Encapsulation\PYZus{}Example}\PY{p}{(}\PY{l+m+mi}{25}\PY{p}{)}\PY{p}{;}
\end{Verbatim}


    \begin{Verbatim}[commandchars=\\\{\}]
{\color{incolor}In [{\color{incolor} }]:} \PY{c+c1}{// what will happen when I run this code?}
        \PY{n}{s}\PY{p}{.}\PY{n}{id\PYZus{}}
\end{Verbatim}


    \hypertarget{aside-initializer-list-construction}{%
\paragraph{Aside: Initializer-list
construction}\label{aside-initializer-list-construction}}

In the above example, we used the syntax

\begin{verbatim}
Student_Encapsulation_Example(int id){
    this->id_ = id;
};
\end{verbatim}

for our constructor. In C++11, the initializer list syntax was
introduced for constructors, so we could have written:

\begin{verbatim}
Student_Encapsulation_Example(int id): id_(id){};
\end{verbatim}

There are various benefits to using this syntax which we will see later
in this lecture. We will also use it for the rest of the examples.

    \begin{Verbatim}[commandchars=\\\{\}]
{\color{incolor}In [{\color{incolor} }]:} \PY{k}{class} \PY{n+nc}{Student\PYZus{}Encapsulation\PYZus{}Example2} \PY{p}{\PYZob{}}
            \PY{c+c1}{// By default, these variables are private}
            \PY{k+kt}{int} \PY{n}{id\PYZus{}}\PY{p}{;}
            \PY{k+kt}{double} \PY{n}{gpa\PYZus{}}\PY{p}{;}
            \PY{n}{std}\PY{o}{:}\PY{o}{:}\PY{n}{vector}\PY{o}{\PYZlt{}}\PY{n}{std}\PY{o}{:}\PY{o}{:}\PY{n}{string}\PY{o}{\PYZgt{}} \PY{n}{courses\PYZus{}}\PY{p}{;}
          
          \PY{k}{public}\PY{o}{:}
            \PY{n}{Student\PYZus{}Encapsulation\PYZus{}Example2}\PY{p}{(}\PY{k+kt}{int} \PY{n}{id}\PY{p}{)}\PY{o}{:} \PY{n}{id\PYZus{}}\PY{p}{(}\PY{n}{id}\PY{p}{)}\PY{p}{\PYZob{}}\PY{p}{\PYZcb{}}\PY{p}{;}
            
            \PY{c+c1}{//This is the default constructor, which takes no arguments}
            \PY{c+c1}{//Members are instantiated to their default values}
            \PY{n}{Student\PYZus{}Encapsulation\PYZus{}Example2}\PY{p}{(}\PY{p}{)}\PY{p}{\PYZob{}}\PY{p}{\PYZcb{}}\PY{p}{;}
            
            \PY{k+kt}{int} \PY{n+nf}{get\PYZus{}id}\PY{p}{(}\PY{p}{)}\PY{p}{\PYZob{}}
                \PY{k}{return} \PY{n}{id\PYZus{}}\PY{p}{;}
            \PY{p}{\PYZcb{}}
        \PY{p}{\PYZcb{}}
\end{Verbatim}


    \begin{Verbatim}[commandchars=\\\{\}]
{\color{incolor}In [{\color{incolor} }]:} \PY{n}{Student\PYZus{}Encapsulation\PYZus{}Example2} \PY{n}{se} \PY{o}{=} \PY{n}{Student\PYZus{}Encapsulation\PYZus{}Example2}\PY{p}{(}\PY{l+m+mi}{25}\PY{p}{)}\PY{p}{;}
\end{Verbatim}


    \begin{Verbatim}[commandchars=\\\{\}]
{\color{incolor}In [{\color{incolor} }]:} \PY{n}{se}\PY{p}{.}\PY{n}{get\PYZus{}id}\PY{p}{(}\PY{p}{)}
\end{Verbatim}


    \begin{Verbatim}[commandchars=\\\{\}]
{\color{incolor}In [{\color{incolor} }]:} \PY{n}{Student\PYZus{}Encapsulation\PYZus{}Example2} \PY{n}{default\PYZus{}se} \PY{o}{=}  \PY{n}{Student\PYZus{}Encapsulation\PYZus{}Example2}\PY{p}{(}\PY{p}{)}\PY{p}{;}
\end{Verbatim}


    \begin{Verbatim}[commandchars=\\\{\}]
{\color{incolor}In [{\color{incolor} }]:} \PY{c+c1}{// What will this return?}
        \PY{n}{default\PYZus{}se}\PY{p}{.}\PY{n}{get\PYZus{}id}\PY{p}{(}\PY{p}{)}
\end{Verbatim}


    \hypertarget{recap-abstraction}{%
\subsubsection{2. Recap: Abstraction}\label{recap-abstraction}}

Abstraction is the principle of hiding unecessary details from other
objects (Closely related to encapsulation). Other objects don't need to
know the details about the inside of another object's class and
\emph{how} its methods work. All that is required is knowledge of what
the methods do, and how to interact with them.

To illustrate this, we will create two new examples of a Student class,
each with the following methods:

\begin{itemize}
\tightlist
\item
  A constructor that takes an integer id as input
\item
  An \texttt{add\_course()} method that takes a string and gradepoint as
  input, and returns nothing
\item
  A \texttt{print\_course\_roster()} method that prints out the current
  roster for the student, and returns nothing
\item
  A \texttt{get\_gpa()} method that returns the student's current gpa
\end{itemize}

    \begin{Verbatim}[commandchars=\\\{\}]
{\color{incolor}In [{\color{incolor} }]:} \PY{k}{class} \PY{n+nc}{Student\PYZus{}Abstraction\PYZus{}Example1} \PY{p}{\PYZob{}}
            \PY{k+kt}{int} \PY{n}{id\PYZus{}}\PY{p}{;}
            \PY{k+kt}{double} \PY{n}{gpa\PYZus{}}\PY{p}{;}
            \PY{n}{std}\PY{o}{:}\PY{o}{:}\PY{n}{vector}\PY{o}{\PYZlt{}}\PY{n}{std}\PY{o}{:}\PY{o}{:}\PY{n}{string}\PY{o}{\PYZgt{}} \PY{n}{courses\PYZus{}}\PY{p}{;}
            \PY{n}{std}\PY{o}{:}\PY{o}{:}\PY{n}{vector}\PY{o}{\PYZlt{}}\PY{k+kt}{double}\PY{o}{\PYZgt{}} \PY{n}{course\PYZus{}grades\PYZus{}}\PY{p}{;}
          
          \PY{k}{public}\PY{o}{:}
            \PY{n}{Student\PYZus{}Abstraction\PYZus{}Example1}\PY{p}{(}\PY{k+kt}{int} \PY{n}{id}\PY{p}{)}\PY{o}{:}\PY{n}{id\PYZus{}}\PY{p}{(}\PY{n}{id}\PY{p}{)}\PY{p}{\PYZob{}}\PY{p}{\PYZcb{}}\PY{p}{;}
            
            \PY{k+kt}{void} \PY{n+nf}{add\PYZus{}course}\PY{p}{(}\PY{n}{std}\PY{o}{:}\PY{o}{:}\PY{n}{string} \PY{n}{name}\PY{p}{,} \PY{k+kt}{double} \PY{n}{gradepoint}\PY{p}{)}\PY{p}{\PYZob{}}
                \PY{n}{courses\PYZus{}}\PY{p}{.}\PY{n}{push\PYZus{}back}\PY{p}{(}\PY{n}{name}\PY{p}{)}\PY{p}{;}
                \PY{n}{course\PYZus{}grades\PYZus{}}\PY{p}{.}\PY{n}{push\PYZus{}back}\PY{p}{(}\PY{n}{gradepoint}\PY{p}{)}\PY{p}{;}
                \PY{k+kt}{double} \PY{n}{sumgradepoints} \PY{o}{=} \PY{n}{std}\PY{o}{:}\PY{o}{:}\PY{n}{accumulate}\PY{p}{(}
                    \PY{n}{course\PYZus{}grades\PYZus{}}\PY{p}{.}\PY{n}{begin}\PY{p}{(}\PY{p}{)}\PY{p}{,} 
                    \PY{n}{course\PYZus{}grades\PYZus{}}\PY{p}{.}\PY{n}{end}\PY{p}{(}\PY{p}{)}\PY{p}{,} 
                    \PY{l+m+mf}{0.0}\PY{p}{)}\PY{p}{;}
                \PY{n}{gpa\PYZus{}} \PY{o}{=} \PY{n}{sumgradepoints} \PY{o}{/} \PY{p}{(}\PY{k+kt}{double}\PY{p}{)}\PY{k}{this}\PY{o}{\PYZhy{}}\PY{o}{\PYZgt{}}\PY{n}{course\PYZus{}grades\PYZus{}}\PY{p}{.}\PY{n}{size}\PY{p}{(}\PY{p}{)}\PY{p}{;}
            \PY{p}{\PYZcb{}}\PY{p}{;}
            
            \PY{k+kt}{void} \PY{n+nf}{print\PYZus{}course\PYZus{}roster}\PY{p}{(}\PY{p}{)}\PY{p}{\PYZob{}}
                \PY{k}{for} \PY{p}{(}\PY{k+kt}{int} \PY{n}{i} \PY{o}{=} \PY{l+m+mi}{0}\PY{p}{;} \PY{n}{i} \PY{o}{\PYZlt{}} \PY{n}{courses\PYZus{}}\PY{p}{.}\PY{n}{size}\PY{p}{(}\PY{p}{)}\PY{p}{;} \PY{n}{i}\PY{o}{+}\PY{o}{+}\PY{p}{)}\PY{p}{\PYZob{}}
                    \PY{n}{std}\PY{o}{:}\PY{o}{:}\PY{n}{cout} \PY{o}{\PYZlt{}}\PY{o}{\PYZlt{}} \PY{n}{courses\PYZus{}}\PY{p}{[}\PY{n}{i}\PY{p}{]} \PY{o}{\PYZlt{}}\PY{o}{\PYZlt{}} \PY{n}{std}\PY{o}{:}\PY{o}{:}\PY{n}{endl}\PY{p}{;}
                \PY{p}{\PYZcb{}}
            \PY{p}{\PYZcb{}}
            
            \PY{k+kt}{double} \PY{n+nf}{get\PYZus{}gpa}\PY{p}{(}\PY{p}{)}\PY{p}{\PYZob{}}
                \PY{k}{return} \PY{n}{gpa\PYZus{}}\PY{p}{;}
            \PY{p}{\PYZcb{}}
        \PY{p}{\PYZcb{}}
\end{Verbatim}


    \begin{Verbatim}[commandchars=\\\{\}]
{\color{incolor}In [{\color{incolor} }]:} \PY{k}{struct} \PY{n}{Student\PYZus{}Course}\PY{p}{\PYZob{}}
            \PY{n}{std}\PY{o}{:}\PY{o}{:}\PY{n}{string} \PY{n}{course\PYZus{}}\PY{p}{;}
            \PY{k+kt}{double} \PY{n}{course\PYZus{}grade\PYZus{}}\PY{p}{;}
            \PY{n}{Student\PYZus{}Course}\PY{p}{(}\PY{n}{std}\PY{o}{:}\PY{o}{:}\PY{n}{string} \PY{n}{course}\PY{p}{,}
                           \PY{k+kt}{double} \PY{n}{grade}\PY{p}{)}\PY{o}{:}
                \PY{n}{course\PYZus{}}\PY{p}{(}\PY{n}{course}\PY{p}{)}\PY{p}{,} \PY{n}{course\PYZus{}grade\PYZus{}}\PY{p}{(}\PY{n}{grade}\PY{p}{)}\PY{p}{\PYZob{}}\PY{p}{\PYZcb{}}\PY{p}{;}
        \PY{p}{\PYZcb{}}\PY{p}{;}
        
        \PY{k}{class} \PY{n+nc}{Student\PYZus{}Abstraction\PYZus{}Example2} \PY{p}{\PYZob{}}
            \PY{k+kt}{int} \PY{n}{id\PYZus{}}\PY{p}{;}
            \PY{k+kt}{double} \PY{n}{gpa\PYZus{}}\PY{p}{;}
            \PY{n}{std}\PY{o}{:}\PY{o}{:}\PY{n}{vector}\PY{o}{\PYZlt{}}\PY{n}{Student\PYZus{}Course}\PY{o}{\PYZgt{}} \PY{n}{courses\PYZus{}}\PY{p}{;}
          
          \PY{k}{public}\PY{o}{:}
            \PY{n}{Student\PYZus{}Abstraction\PYZus{}Example2}\PY{p}{(}\PY{k+kt}{int} \PY{n}{id}\PY{p}{)}\PY{o}{:}\PY{n}{id\PYZus{}}\PY{p}{(}\PY{n}{id}\PY{p}{)}\PY{p}{\PYZob{}}\PY{p}{\PYZcb{}}\PY{p}{;}
            
            \PY{k+kt}{void} \PY{n+nf}{add\PYZus{}course}\PY{p}{(}\PY{n}{std}\PY{o}{:}\PY{o}{:}\PY{n}{string} \PY{n}{name}\PY{p}{,} \PY{k+kt}{double} \PY{n}{gradepoint}\PY{p}{)}\PY{p}{\PYZob{}}
                \PY{n}{courses\PYZus{}}\PY{p}{.}\PY{n}{emplace\PYZus{}back}\PY{p}{(}\PY{n}{name}\PY{p}{,}\PY{n}{gradepoint}\PY{p}{)}\PY{p}{;}
                \PY{k+kt}{double} \PY{n}{sumgradepoints} \PY{o}{=} 
                    \PY{n}{std}\PY{o}{:}\PY{o}{:}\PY{n}{accumulate}\PY{p}{(}\PY{n}{courses\PYZus{}}\PY{p}{.}\PY{n}{begin}\PY{p}{(}\PY{p}{)}\PY{p}{,}
                                    \PY{n}{courses\PYZus{}}\PY{p}{.}\PY{n}{end}\PY{p}{(}\PY{p}{)}\PY{p}{,}
                                    \PY{l+m+mf}{0.0}\PY{p}{,}
                                    \PY{p}{[}\PY{p}{]}\PY{p}{(}\PY{k+kt}{double} \PY{n}{curr\PYZus{}sum}\PY{p}{,} \PY{n}{Student\PYZus{}Course} \PY{n}{course}\PY{p}{)}\PY{p}{\PYZob{}}
                                        \PY{k}{return} \PY{n}{curr\PYZus{}sum} \PY{o}{+} \PY{n}{course}\PY{p}{.}\PY{n}{course\PYZus{}grade\PYZus{}}\PY{p}{;}\PY{p}{\PYZcb{}}
                                   \PY{p}{)}\PY{p}{;}
                \PY{n}{gpa\PYZus{}} \PY{o}{=} \PY{n}{sumgradepoints} \PY{o}{/} \PY{p}{(}\PY{k+kt}{double}\PY{p}{)}\PY{n}{courses\PYZus{}}\PY{p}{.}\PY{n}{size}\PY{p}{(}\PY{p}{)}\PY{p}{;}
            \PY{p}{\PYZcb{}}
            
            \PY{k+kt}{void} \PY{n+nf}{print\PYZus{}course\PYZus{}roster}\PY{p}{(}\PY{p}{)}\PY{p}{\PYZob{}}
                \PY{k}{for} \PY{p}{(}\PY{k+kt}{int} \PY{n}{i} \PY{o}{=} \PY{l+m+mi}{0}\PY{p}{;} \PY{n}{i} \PY{o}{\PYZlt{}} \PY{n}{courses\PYZus{}}\PY{p}{.}\PY{n}{size}\PY{p}{(}\PY{p}{)}\PY{p}{;} \PY{n}{i}\PY{o}{+}\PY{o}{+}\PY{p}{)}\PY{p}{\PYZob{}}
                    \PY{n}{std}\PY{o}{:}\PY{o}{:}\PY{n}{cout} \PY{o}{\PYZlt{}}\PY{o}{\PYZlt{}} \PY{n}{courses\PYZus{}}\PY{p}{[}\PY{n}{i}\PY{p}{]}\PY{p}{.}\PY{n}{course\PYZus{}} \PY{o}{\PYZlt{}}\PY{o}{\PYZlt{}} \PY{n}{std}\PY{o}{:}\PY{o}{:}\PY{n}{endl}\PY{p}{;}
                \PY{p}{\PYZcb{}}
            \PY{p}{\PYZcb{}}
            
            \PY{k+kt}{double} \PY{n+nf}{get\PYZus{}gpa}\PY{p}{(}\PY{p}{)}\PY{p}{\PYZob{}}
                \PY{k}{return} \PY{n}{gpa\PYZus{}}\PY{p}{;}
            \PY{p}{\PYZcb{}}
            
        \PY{p}{\PYZcb{}}
\end{Verbatim}


    \begin{Verbatim}[commandchars=\\\{\}]
{\color{incolor}In [{\color{incolor} }]:} \PY{n}{Student\PYZus{}Abstraction\PYZus{}Example1} \PY{n}{s1} \PY{o}{=} \PY{n}{Student\PYZus{}Abstraction\PYZus{}Example1}\PY{p}{(}\PY{l+m+mi}{3}\PY{p}{)}\PY{p}{;}
        \PY{n}{s1}\PY{p}{.}\PY{n}{add\PYZus{}course}\PY{p}{(}\PY{l+s}{\PYZdq{}}\PY{l+s}{CME 211}\PY{l+s}{\PYZdq{}}\PY{p}{,} \PY{l+m+mf}{3.4}\PY{p}{)}\PY{p}{;}
        \PY{n}{s1}\PY{p}{.}\PY{n}{print\PYZus{}course\PYZus{}roster}\PY{p}{(}\PY{p}{)}\PY{p}{;}
        
        \PY{n}{Student\PYZus{}Abstraction\PYZus{}Example2} \PY{n}{s2} \PY{o}{=} \PY{n}{Student\PYZus{}Abstraction\PYZus{}Example2}\PY{p}{(}\PY{l+m+mi}{3}\PY{p}{)}\PY{p}{;}
        \PY{n}{s2}\PY{p}{.}\PY{n}{add\PYZus{}course}\PY{p}{(}\PY{l+s}{\PYZdq{}}\PY{l+s}{CME 211}\PY{l+s}{\PYZdq{}}\PY{p}{,} \PY{l+m+mf}{3.4}\PY{p}{)}\PY{p}{;}
        \PY{n}{s2}\PY{p}{.}\PY{n}{print\PYZus{}course\PYZus{}roster}\PY{p}{(}\PY{p}{)}\PY{p}{;}
\end{Verbatim}


    \begin{Verbatim}[commandchars=\\\{\}]
{\color{incolor}In [{\color{incolor} }]:} \PY{n}{s1}\PY{p}{.}\PY{n}{get\PYZus{}gpa}\PY{p}{(}\PY{p}{)}
\end{Verbatim}


    \begin{Verbatim}[commandchars=\\\{\}]
{\color{incolor}In [{\color{incolor} }]:} \PY{n}{s2}\PY{p}{.}\PY{n}{get\PYZus{}gpa}\PY{p}{(}\PY{p}{)}
\end{Verbatim}


    To a user of this code, the two Student classes function identically,
though their internal implementations of the course roster are
different.

How would a C++ developer communicate to users the functionality of the
Student class? One way is through the header file for the student class,
which would contain:

    \emph{file: Student.hpp}

\begin{verbatim}
class Student {
    Student(int id);
    void add_course(std::string name, double gradepoint);
    void print_course_roster();
    double get_gpa();
}
\end{verbatim}

    \emph{file: Student.cpp}

\begin{verbatim}
#include <vector>
#include <iostream>

#include "Student.hpp"

struct Student_Course{
    std::string course_;
    double course_grade_;
    Student_Course(std::string course, double grade):course_(course), course_grade_(grade){};
};

class Student{
    int id_;
    double gpa_;
    std::vector<Student_Course> courses_;
  
  public:
    Student(int id):id_(id){};
    
    void add_course(std::string name, double gradepoint){
        courses_.emplace_back(name,gradepoint);
        double sumgradepoints = std::accumulate(courses_.begin(),
                                               courses_.end(),
                                               0.0,
                                               [](double curr_sum, Student_Course course){
                                                   return curr_sum + course.course_grade_;}
                                              );
        gpa_ = sumgradepoints / (double)courses_.size();
    }
    
    void print_course_roster(){
        for (int i = 0; i < courses_.size(); i++){
            std::cout << courses_[i].course_ << std::endl;
        }
    }
    
    double get_gpa(){
        return gpa_;
    }
    
}
\end{verbatim}

    What if instead of just changing the implementation details of our
Student class, we actually wanted to create different kinds of Students?

For example: Suppose that Students could either be SPCD or live on
campus, and we wanted the Student object to have different functionality
based on this distinction.

This leads us to the principle of \textbf{Inheritance}.

\hypertarget{inheritance}{%
\subsubsection{3. Inheritance}\label{inheritance}}

    \begin{Verbatim}[commandchars=\\\{\}]
{\color{incolor}In [{\color{incolor} }]:} \PY{c+cp}{\PYZsh{}}\PY{c+cp}{include} \PY{c+cpf}{\PYZlt{}iostream\PYZgt{}}
\end{Verbatim}


    \begin{Verbatim}[commandchars=\\\{\}]
{\color{incolor}In [{\color{incolor} }]:} \PY{k}{class} \PY{n+nc}{Student} \PY{p}{\PYZob{}}
            \PY{k+kt}{int} \PY{n}{id\PYZus{}}\PY{p}{;}
            \PY{k+kt}{double} \PY{n}{gpa\PYZus{}}\PY{p}{;}
            \PY{n}{std}\PY{o}{:}\PY{o}{:}\PY{n}{vector}\PY{o}{\PYZlt{}}\PY{n}{std}\PY{o}{:}\PY{o}{:}\PY{n}{string}\PY{o}{\PYZgt{}} \PY{n}{courses\PYZus{}}\PY{p}{;}
            \PY{n}{std}\PY{o}{:}\PY{o}{:}\PY{n}{vector}\PY{o}{\PYZlt{}}\PY{k+kt}{double}\PY{o}{\PYZgt{}} \PY{n}{course\PYZus{}grades\PYZus{}}\PY{p}{;}
          
          \PY{k}{public}\PY{o}{:}
            \PY{n}{Student}\PY{p}{(}\PY{k+kt}{int} \PY{n}{id}\PY{p}{)}\PY{o}{:}\PY{n}{id\PYZus{}}\PY{p}{(}\PY{n}{id}\PY{p}{)}\PY{p}{\PYZob{}}\PY{p}{\PYZcb{}}\PY{p}{;}
            
            \PY{k+kt}{void} \PY{n+nf}{add\PYZus{}course}\PY{p}{(}\PY{n}{std}\PY{o}{:}\PY{o}{:}\PY{n}{string} \PY{n}{name}\PY{p}{,} \PY{k+kt}{double} \PY{n}{gradepoint}\PY{p}{)}\PY{p}{\PYZob{}}
                \PY{n}{courses\PYZus{}}\PY{p}{.}\PY{n}{push\PYZus{}back}\PY{p}{(}\PY{n}{name}\PY{p}{)}\PY{p}{;}
                \PY{n}{course\PYZus{}grades\PYZus{}}\PY{p}{.}\PY{n}{push\PYZus{}back}\PY{p}{(}\PY{n}{gradepoint}\PY{p}{)}\PY{p}{;}
                \PY{k+kt}{double} \PY{n}{sumgradepoints} \PY{o}{=} \PY{n}{std}\PY{o}{:}\PY{o}{:}\PY{n}{accumulate}\PY{p}{(}
                    \PY{n}{course\PYZus{}grades\PYZus{}}\PY{p}{.}\PY{n}{begin}\PY{p}{(}\PY{p}{)}\PY{p}{,} 
                    \PY{n}{course\PYZus{}grades\PYZus{}}\PY{p}{.}\PY{n}{end}\PY{p}{(}\PY{p}{)}\PY{p}{,} 
                    \PY{l+m+mf}{0.0}\PY{p}{)}\PY{p}{;}
                \PY{n}{gpa\PYZus{}} \PY{o}{=} \PY{n}{sumgradepoints} \PY{o}{/} \PY{p}{(}\PY{k+kt}{double}\PY{p}{)}\PY{n}{course\PYZus{}grades\PYZus{}}\PY{p}{.}\PY{n}{size}\PY{p}{(}\PY{p}{)}\PY{p}{;}
            \PY{p}{\PYZcb{}}\PY{p}{;}
            
            \PY{k}{const} \PY{k+kt}{void} \PY{n+nf}{print\PYZus{}course\PYZus{}roster}\PY{p}{(}\PY{p}{)}\PY{p}{\PYZob{}}
                \PY{k}{for} \PY{p}{(}\PY{k+kt}{int} \PY{n}{i} \PY{o}{=} \PY{l+m+mi}{0}\PY{p}{;} \PY{n}{i} \PY{o}{\PYZlt{}} \PY{n}{courses\PYZus{}}\PY{p}{.}\PY{n}{size}\PY{p}{(}\PY{p}{)}\PY{p}{;} \PY{n}{i}\PY{o}{+}\PY{o}{+}\PY{p}{)}\PY{p}{\PYZob{}}
                    \PY{n}{std}\PY{o}{:}\PY{o}{:}\PY{n}{cout} \PY{o}{\PYZlt{}}\PY{o}{\PYZlt{}} \PY{n}{courses\PYZus{}}\PY{p}{[}\PY{n}{i}\PY{p}{]} \PY{o}{\PYZlt{}}\PY{o}{\PYZlt{}} \PY{n}{std}\PY{o}{:}\PY{o}{:}\PY{n}{endl}\PY{p}{;}
                \PY{p}{\PYZcb{}}
            \PY{p}{\PYZcb{}}
            
            \PY{k}{const} \PY{k+kt}{double} \PY{n+nf}{get\PYZus{}gpa}\PY{p}{(}\PY{p}{)}\PY{p}{\PYZob{}}
                \PY{k}{return} \PY{n}{gpa\PYZus{}}\PY{p}{;}
            \PY{p}{\PYZcb{}}
            
            \PY{k}{const} \PY{k+kt}{double} \PY{n+nf}{get\PYZus{}id}\PY{p}{(}\PY{p}{)}\PY{p}{\PYZob{}}
                \PY{k}{return} \PY{n}{id\PYZus{}}\PY{p}{;}
            \PY{p}{\PYZcb{}}
            
            \PY{k}{virtual} \PY{n}{std}\PY{o}{:}\PY{o}{:}\PY{n}{string} \PY{n}{get\PYZus{}dorm}\PY{p}{(}\PY{p}{)}\PY{p}{\PYZob{}}
                \PY{k}{return} \PY{l+s}{\PYZdq{}}\PY{l+s}{No dorm assigned}\PY{l+s}{\PYZdq{}}\PY{p}{;}
            \PY{p}{\PYZcb{}}
            
        \PY{p}{\PYZcb{}}
\end{Verbatim}


    \begin{Verbatim}[commandchars=\\\{\}]
{\color{incolor}In [{\color{incolor} }]:} \PY{k}{class} \PY{n+nc}{SCPD\PYZus{}Student} \PY{o}{:} \PY{k}{public} \PY{n}{Student} \PY{p}{\PYZob{}}
            \PY{n}{std}\PY{o}{:}\PY{o}{:}\PY{n}{string} \PY{n}{location\PYZus{}}\PY{p}{;}
          \PY{k}{public}\PY{o}{:}
            \PY{c+c1}{// Note that the constructor for Student }
            \PY{c+c1}{// is explicitly called with the parameter (\PYZdq{}id\PYZdq{})}
            \PY{c+c1}{// To pass a parameter to the parent, }
            \PY{c+c1}{// we must use the initializer list construction}
            \PY{c+c1}{// If we don\PYZsq{}t explicitly call the parent constructor,}
            \PY{c+c1}{// then the default parent constructor is called}
            \PY{n}{SCPD\PYZus{}Student}\PY{p}{(}\PY{k+kt}{int} \PY{n}{id}\PY{p}{,} \PY{n}{std}\PY{o}{:}\PY{o}{:}\PY{n}{string} \PY{n}{location}\PY{p}{)} \PY{o}{:} \PY{n}{Student}\PY{p}{(}\PY{n}{id}\PY{p}{)}\PY{p}{,} \PY{n}{location\PYZus{}}\PY{p}{(}\PY{n}{location}\PY{p}{)} \PY{p}{\PYZob{}}\PY{p}{\PYZcb{}}\PY{p}{;}
            
            \PY{k}{const} \PY{n}{std}\PY{o}{:}\PY{o}{:}\PY{n}{string} \PY{n}{get\PYZus{}location}\PY{p}{(}\PY{p}{)}\PY{p}{\PYZob{}}
                \PY{k}{return} \PY{n}{location\PYZus{}}\PY{p}{;}
            \PY{p}{\PYZcb{}}
        
        \PY{p}{\PYZcb{}}
\end{Verbatim}


    \begin{Verbatim}[commandchars=\\\{\}]
{\color{incolor}In [{\color{incolor} }]:} \PY{k}{class} \PY{n+nc}{Local\PYZus{}Student} \PY{o}{:} \PY{k}{public} \PY{n}{Student} \PY{p}{\PYZob{}}
            \PY{n}{std}\PY{o}{:}\PY{o}{:}\PY{n}{string} \PY{n}{dorm\PYZus{}}\PY{p}{;}
          \PY{k}{public}\PY{o}{:}
            \PY{n}{Local\PYZus{}Student}\PY{p}{(}\PY{k+kt}{int} \PY{n}{id}\PY{p}{,} \PY{n}{std}\PY{o}{:}\PY{o}{:}\PY{n}{string} \PY{n}{dorm}\PY{p}{)} \PY{o}{:} \PY{n}{Student}\PY{p}{(}\PY{n}{id}\PY{p}{)}\PY{p}{,} \PY{n}{dorm\PYZus{}}\PY{p}{(}\PY{n}{dorm}\PY{p}{)} \PY{p}{\PYZob{}}\PY{p}{\PYZcb{}}\PY{p}{;}
            
            \PY{n}{std}\PY{o}{:}\PY{o}{:}\PY{n}{string} \PY{n}{get\PYZus{}dorm}\PY{p}{(}\PY{p}{)}\PY{p}{\PYZob{}}
                \PY{k}{return} \PY{n}{dorm\PYZus{}}\PY{p}{;}
            \PY{p}{\PYZcb{}}
            
        \PY{p}{\PYZcb{}}
\end{Verbatim}


    \begin{Verbatim}[commandchars=\\\{\}]
{\color{incolor}In [{\color{incolor} }]:} \PY{n}{SCPD\PYZus{}Student} \PY{n}{remote\PYZus{}student} \PY{o}{=} \PY{n}{SCPD\PYZus{}Student}\PY{p}{(}\PY{l+m+mi}{34}\PY{p}{,} \PY{l+s}{\PYZdq{}}\PY{l+s}{Minneapolis}\PY{l+s}{\PYZdq{}}\PY{p}{)}\PY{p}{;}
        \PY{n}{Local\PYZus{}Student} \PY{n}{local} \PY{o}{=} \PY{n}{Local\PYZus{}Student}\PY{p}{(}\PY{l+m+mi}{25}\PY{p}{,} \PY{l+s}{\PYZdq{}}\PY{l+s}{Lyman}\PY{l+s}{\PYZdq{}}\PY{p}{)}\PY{p}{;}
\end{Verbatim}


    \begin{Verbatim}[commandchars=\\\{\}]
{\color{incolor}In [{\color{incolor} }]:} \PY{c+c1}{// Parent methods are inherited by children automatically}
        \PY{n}{local}\PY{p}{.}\PY{n}{get\PYZus{}gpa}\PY{p}{(}\PY{p}{)}
\end{Verbatim}


    \begin{Verbatim}[commandchars=\\\{\}]
{\color{incolor}In [{\color{incolor} }]:} \PY{c+c1}{// Methods defined in the child class are also accessible}
        \PY{n}{remote\PYZus{}student}\PY{p}{.}\PY{n}{get\PYZus{}location}\PY{p}{(}\PY{p}{)}
\end{Verbatim}


    \begin{Verbatim}[commandchars=\\\{\}]
{\color{incolor}In [{\color{incolor} }]:} \PY{c+c1}{// What about methods defined in a sibling class?}
        \PY{n}{local}\PY{p}{.}\PY{n}{get\PYZus{}location}\PY{p}{(}\PY{p}{)}
\end{Verbatim}


    \hypertarget{virtual-methods}{%
\paragraph{3.1 Virtual methods}\label{virtual-methods}}

Notice that we used the \textbf{virtual} keyword in the Student class
when defining our method \texttt{get\_dorm()}

\begin{verbatim}
    virtual std::string get_dorm(){
        return "No dorm assigned";
    }
\end{verbatim}

This tells the compiler that the function can be overridden in a derived
class, though it doesn't have to be.

    \begin{Verbatim}[commandchars=\\\{\}]
{\color{incolor}In [{\color{incolor} }]:} \PY{n}{local}\PY{p}{.}\PY{n}{get\PYZus{}dorm}\PY{p}{(}\PY{p}{)}
\end{Verbatim}


    \begin{Verbatim}[commandchars=\\\{\}]
{\color{incolor}In [{\color{incolor} }]:} \PY{n}{remote\PYZus{}student}\PY{p}{.}\PY{n}{get\PYZus{}dorm}\PY{p}{(}\PY{p}{)}
\end{Verbatim}


    Let's take a closer look at the syntax that we used to establish the
inheritance relationship:
\texttt{class\ Local\_Student\ :\ public\ Student}

Note that we used the \textbf{public} keyword. This meant that all of
the \texttt{public} and \texttt{protected} members and methods of the
Student class were also public in the Local\_Student class, which is why
we were able to call \texttt{get\_gpa()}.

\textbf{private} inheritance is also an option, though less common.

What is so useful about establishing inheritance relationships this way?
\textbf{Polymorphism}

    \hypertarget{polymorphism}{%
\subsubsection{4. Polymorphism}\label{polymorphism}}

The concept that a different version of a method can be called based on
the inheritance structure of the classes. This allows us to interact
with ``Student'' objects whose underlying functionality is dictated by
their actual type.

    \begin{Verbatim}[commandchars=\\\{\}]
{\color{incolor}In [{\color{incolor} }]:} \PY{n}{std}\PY{o}{:}\PY{o}{:}\PY{n}{vector}\PY{o}{\PYZlt{}}\PY{n}{Student}\PY{o}{*}\PY{o}{\PYZgt{}} \PY{n}{students\PYZus{}}\PY{p}{;}
\end{Verbatim}


    \begin{Verbatim}[commandchars=\\\{\}]
{\color{incolor}In [{\color{incolor} }]:} \PY{n}{students\PYZus{}}\PY{p}{.}\PY{n}{push\PYZus{}back}\PY{p}{(}\PY{o}{\PYZam{}}\PY{n}{local}\PY{p}{)}\PY{p}{;}
\end{Verbatim}


    \begin{Verbatim}[commandchars=\\\{\}]
{\color{incolor}In [{\color{incolor} }]:} \PY{n}{students\PYZus{}}\PY{p}{.}\PY{n}{push\PYZus{}back}\PY{p}{(}\PY{o}{\PYZam{}}\PY{n}{remote\PYZus{}student}\PY{p}{)}\PY{p}{;}
\end{Verbatim}


    \begin{Verbatim}[commandchars=\\\{\}]
{\color{incolor}In [{\color{incolor} }]:} \PY{k}{for} \PY{p}{(}\PY{k+kt}{int} \PY{n}{i} \PY{o}{=} \PY{l+m+mi}{0}\PY{p}{;} \PY{n}{i} \PY{o}{\PYZlt{}} \PY{n}{students\PYZus{}}\PY{p}{.}\PY{n}{size}\PY{p}{(}\PY{p}{)}\PY{p}{;} \PY{n}{i}\PY{o}{+}\PY{o}{+}\PY{p}{)}\PY{p}{\PYZob{}}
            \PY{n}{std}\PY{o}{:}\PY{o}{:}\PY{n}{cout} \PY{o}{\PYZlt{}}\PY{o}{\PYZlt{}} \PY{l+s}{\PYZdq{}}\PY{l+s}{Student }\PY{l+s}{\PYZdq{}} \PY{o}{\PYZlt{}}\PY{o}{\PYZlt{}} \PY{n}{students\PYZus{}}\PY{p}{[}\PY{n}{i}\PY{p}{]}\PY{o}{\PYZhy{}}\PY{o}{\PYZgt{}}\PY{n}{get\PYZus{}id}\PY{p}{(}\PY{p}{)} \PY{o}{\PYZlt{}}\PY{o}{\PYZlt{}} \PY{l+s}{\PYZdq{}}\PY{l+s}{: }\PY{l+s}{\PYZdq{}} 
                \PY{o}{\PYZlt{}}\PY{o}{\PYZlt{}} \PY{n}{students\PYZus{}}\PY{p}{[}\PY{n}{i}\PY{p}{]}\PY{o}{\PYZhy{}}\PY{o}{\PYZgt{}}\PY{n}{get\PYZus{}dorm}\PY{p}{(}\PY{p}{)} \PY{o}{\PYZlt{}}\PY{o}{\PYZlt{}} \PY{n}{std}\PY{o}{:}\PY{o}{:}\PY{n}{endl}\PY{p}{;}
        \PY{p}{\PYZcb{}}
\end{Verbatim}


    Note that if we hadn't included the \texttt{virtual} keyword, then the
base class's version of \texttt{get\_dorm()} would have been called,
even for the local student.

The \texttt{virtual} keyword signals to the compiler that we don't want
\textbf{static linkage} for this function (function call determined
before the program is executed).

Intead, we want the selection of which version of \texttt{get\_dorm()}
to call to be dictated by the kind of object for which it is called -
this is called \textbf{dynamic linkage} or late binding.

    \hypertarget{abstract-classes}{%
\subsubsection{5. Abstract Classes}\label{abstract-classes}}

Based on the way we defined our Student class so far, we can still
instantiate it (Create objects of type ``Student'')

    \begin{Verbatim}[commandchars=\\\{\}]
{\color{incolor}In [{\color{incolor} }]:} \PY{n}{Student} \PY{n}{base} \PY{o}{=} \PY{n}{Student}\PY{p}{(}\PY{l+m+mi}{22}\PY{p}{)}\PY{p}{;}
\end{Verbatim}


    \begin{Verbatim}[commandchars=\\\{\}]
{\color{incolor}In [{\color{incolor} }]:} \PY{n}{base}\PY{p}{.}\PY{n}{get\PYZus{}dorm}\PY{p}{(}\PY{p}{)}
\end{Verbatim}


    What if we wanted to prevent people from creating a Student object on
its own, and force all students to belong to one of the child classes
(either Local\_Students or SCPD\_Students)? Then, we would want to
create an \textbf{abstract class} - a class that specifies some of the
functionality of its children, but cannot be instantiated.

We will illustrate this by moving on to another example from the Python
OOP lecture.

    \hypertarget{example}{%
\paragraph{Example}\label{example}}

    \begin{verbatim}
import math

class Shape:
    def GetArea(self):
        raise RuntimeError("Not implemented yet")

class Circle(Shape):
    def __init__ (self, x, y, radius):
        self.x = x
        self.y = y
        self.radius = radius

    def GetArea(self):
        area = math.pi * math.pow(self.radius, 2)
        return area

class Rectangle(Shape):
    def __init__ (self, x0, y0, x1, y1):
        self.x0 = x0
        self.y0 = y0
        self.x1 = x1
        self.y1 = y1

    def GetArea(self):
        xDistance = self.x1 - self.x0
        yDistance = self.y1 - self.y0
        return abs(xDistance * yDistance)
\end{verbatim}

    Recall - in this example, Shape is an abstract class which cannot be
instantiated. Here is the same code, but implemented in C++:

    \begin{Verbatim}[commandchars=\\\{\}]
{\color{incolor}In [{\color{incolor} }]:} \PY{k}{class} \PY{n+nc}{Shape} \PY{p}{\PYZob{}}
          \PY{k}{public}\PY{o}{:}
            \PY{c+c1}{//Notice, the virtual keyword and \PYZdq{}= 0\PYZdq{} }
            \PY{k}{virtual} \PY{k+kt}{double} \PY{n}{GetArea}\PY{p}{(}\PY{p}{)} \PY{o}{=} \PY{l+m+mi}{0}\PY{p}{;}
        \PY{p}{\PYZcb{}}
\end{Verbatim}


    \begin{Verbatim}[commandchars=\\\{\}]
{\color{incolor}In [{\color{incolor} }]:} \PY{c+cp}{\PYZsh{}}\PY{c+cp}{include} \PY{c+cpf}{\PYZlt{}math.h\PYZgt{}}
        
        \PY{k}{class} \PY{n+nc}{Circle}\PY{o}{:} \PY{k}{public} \PY{n}{Shape} \PY{p}{\PYZob{}}
            \PY{k+kt}{double} \PY{n}{x\PYZus{}}\PY{p}{;}
            \PY{k+kt}{double} \PY{n}{y\PYZus{}}\PY{p}{;}
            \PY{k+kt}{double} \PY{n}{radius\PYZus{}}\PY{p}{;}
            
          \PY{k}{public}\PY{o}{:}
            \PY{n}{Circle}\PY{p}{(}\PY{k+kt}{double} \PY{n}{x}\PY{p}{,} 
                   \PY{k+kt}{double} \PY{n}{y}\PY{p}{,} 
                   \PY{k+kt}{double} \PY{n}{radius}\PY{p}{)}\PY{o}{:}
                \PY{n}{x\PYZus{}}\PY{p}{(}\PY{n}{x}\PY{p}{)}\PY{p}{,} \PY{n}{y\PYZus{}}\PY{p}{(}\PY{n}{y}\PY{p}{)}\PY{p}{,} \PY{n}{radius\PYZus{}}\PY{p}{(}\PY{n}{radius}\PY{p}{)}\PY{p}{\PYZob{}}\PY{p}{\PYZcb{}}\PY{p}{;}
            
            \PY{k+kt}{double} \PY{n+nf}{GetArea}\PY{p}{(}\PY{p}{)}\PY{p}{\PYZob{}}
                \PY{k}{return} \PY{n}{M\PYZus{}PI} \PY{o}{*} \PY{n}{radius\PYZus{}} \PY{o}{*} \PY{n}{radius\PYZus{}}\PY{p}{;}
            \PY{p}{\PYZcb{}}\PY{p}{;}
            
        \PY{p}{\PYZcb{}}
\end{Verbatim}


    \begin{Verbatim}[commandchars=\\\{\}]
{\color{incolor}In [{\color{incolor} }]:} \PY{k}{class} \PY{n+nc}{Rectangle}\PY{o}{:} \PY{k}{public} \PY{n}{Shape} \PY{p}{\PYZob{}}
            \PY{k+kt}{double} \PY{n}{x0\PYZus{}}\PY{p}{;}
            \PY{k+kt}{double} \PY{n}{y0\PYZus{}}\PY{p}{;}
            \PY{k+kt}{double} \PY{n}{x1\PYZus{}}\PY{p}{;}
            \PY{k+kt}{double} \PY{n}{y1\PYZus{}}\PY{p}{;}
            
          \PY{k}{public}\PY{o}{:}
            \PY{n}{Rectangle}\PY{p}{(}\PY{k+kt}{double} \PY{n}{x0}\PY{p}{,} 
                      \PY{k+kt}{double} \PY{n}{y0}\PY{p}{,} 
                      \PY{k+kt}{double} \PY{n}{x1}\PY{p}{,} 
                      \PY{k+kt}{double} \PY{n}{y1}\PY{p}{)}\PY{o}{:}
            \PY{n}{x0\PYZus{}}\PY{p}{(}\PY{n}{x0}\PY{p}{)}\PY{p}{,} \PY{n}{y0\PYZus{}}\PY{p}{(}\PY{n}{y0}\PY{p}{)}\PY{p}{,} \PY{n}{x1\PYZus{}}\PY{p}{(}\PY{n}{x1}\PY{p}{)}\PY{p}{,} \PY{n}{y1\PYZus{}}\PY{p}{(}\PY{n}{y1}\PY{p}{)}\PY{p}{\PYZob{}}\PY{p}{\PYZcb{}}\PY{p}{;} 
            
            \PY{k+kt}{double} \PY{n+nf}{GetArea}\PY{p}{(}\PY{p}{)}\PY{p}{\PYZob{}}
                \PY{k}{return} \PY{n}{abs}\PY{p}{(}\PY{n}{x0\PYZus{}} \PY{o}{\PYZhy{}} \PY{n}{x1\PYZus{}}\PY{p}{)} \PY{o}{*} \PY{n}{abs}\PY{p}{(}\PY{n}{y0\PYZus{}} \PY{o}{\PYZhy{}} \PY{n}{y1\PYZus{}}\PY{p}{)}\PY{p}{;}
            \PY{p}{\PYZcb{}}
            
        \PY{p}{\PYZcb{}}
\end{Verbatim}


    \begin{Verbatim}[commandchars=\\\{\}]
{\color{incolor}In [{\color{incolor} }]:} \PY{c+c1}{// First, what happens if we try to instantiate Shape?}
        \PY{n}{Shape} \PY{n}{shape} \PY{o}{=} \PY{n}{Shape}\PY{p}{(}\PY{p}{)}\PY{p}{;}
\end{Verbatim}


    \begin{Verbatim}[commandchars=\\\{\}]
{\color{incolor}In [{\color{incolor} }]:} \PY{n}{Rectangle} \PY{n}{rect} \PY{o}{=} \PY{n}{Rectangle}\PY{p}{(}\PY{l+m+mi}{0}\PY{p}{,}\PY{l+m+mi}{0}\PY{p}{,}\PY{l+m+mi}{2}\PY{p}{,}\PY{l+m+mi}{5}\PY{p}{)}\PY{p}{;}
\end{Verbatim}


    \begin{Verbatim}[commandchars=\\\{\}]
{\color{incolor}In [{\color{incolor} }]:} \PY{n}{rect}\PY{p}{.}\PY{n}{GetArea}\PY{p}{(}\PY{p}{)}
\end{Verbatim}


    \begin{Verbatim}[commandchars=\\\{\}]
{\color{incolor}In [{\color{incolor} }]:} \PY{n}{Circle} \PY{n}{circ} \PY{o}{=} \PY{n}{Circle}\PY{p}{(}\PY{l+m+mi}{0}\PY{p}{,}\PY{l+m+mi}{0}\PY{p}{,}\PY{l+m+mi}{6}\PY{p}{)}\PY{p}{;}
\end{Verbatim}


    \begin{Verbatim}[commandchars=\\\{\}]
{\color{incolor}In [{\color{incolor} }]:} \PY{n}{circ}\PY{p}{.}\PY{n}{GetArea}\PY{p}{(}\PY{p}{)}
\end{Verbatim}


    \hypertarget{pure-virtual-methods}{%
\paragraph{5.1 Pure virtual methods}\label{pure-virtual-methods}}

Recap: What did we just observe?

\begin{itemize}
\tightlist
\item
  We declared a function \texttt{virtual\ double\ GetArea()\ =\ 0;} in
  the Shape class. The =0 syntax told the compiler that this was a
  \textbf{pure virtual} function, meaning that it cannot be executed in
  the base class.
\item
  Any class with \textgreater{}= 1 pure virtual function is understood
  to be an \textbf{abstract class} in C++, meaning that it cannot be
  instantiated.
\end{itemize}

    \textbf{Q: Does the concept of polymorphism still apply even with an
abstract class, such as ``Shape''?} A: Yes. It is still valid to have a
pointer of type Shape.

    \begin{Verbatim}[commandchars=\\\{\}]
{\color{incolor}In [{\color{incolor} }]:} \PY{n}{std}\PY{o}{:}\PY{o}{:}\PY{n}{vector}\PY{o}{\PYZlt{}}\PY{n}{Shape}\PY{o}{*}\PY{o}{\PYZgt{}} \PY{n}{my\PYZus{}vector}\PY{p}{;}
        
        \PY{n}{my\PYZus{}vector}\PY{p}{.}\PY{n}{emplace\PYZus{}back}\PY{p}{(}\PY{o}{\PYZam{}}\PY{n}{circ}\PY{p}{)}\PY{p}{;}
        \PY{n}{my\PYZus{}vector}\PY{p}{.}\PY{n}{emplace\PYZus{}back}\PY{p}{(}\PY{o}{\PYZam{}}\PY{n}{rect}\PY{p}{)}\PY{p}{;}
\end{Verbatim}


    \begin{Verbatim}[commandchars=\\\{\}]
{\color{incolor}In [{\color{incolor} }]:} \PY{k+kt}{double} \PY{n}{total\PYZus{}area} \PY{o}{=} \PY{l+m+mf}{0.0}\PY{p}{;}
        \PY{k}{for} \PY{p}{(}\PY{k+kt}{int} \PY{n}{i} \PY{o}{=} \PY{l+m+mi}{0}\PY{p}{;} \PY{n}{i} \PY{o}{\PYZlt{}} \PY{n}{my\PYZus{}vector}\PY{p}{.}\PY{n}{size}\PY{p}{(}\PY{p}{)}\PY{p}{;} \PY{n}{i}\PY{o}{+}\PY{o}{+}\PY{p}{)}\PY{p}{\PYZob{}}
            \PY{n}{total\PYZus{}area} \PY{o}{+}\PY{o}{=} \PY{n}{my\PYZus{}vector}\PY{p}{[}\PY{n}{i}\PY{p}{]}\PY{o}{\PYZhy{}}\PY{o}{\PYZgt{}}\PY{n}{GetArea}\PY{p}{(}\PY{p}{)}\PY{p}{;}
        \PY{p}{\PYZcb{}}
\end{Verbatim}


    \begin{Verbatim}[commandchars=\\\{\}]
{\color{incolor}In [{\color{incolor} }]:} \PY{n}{total\PYZus{}area}
\end{Verbatim}


    \hypertarget{composition}{%
\subsubsection{Composition}\label{composition}}

    Composition is another type of relationship between objects. Composition
is when objects relate in a ``has a'' relationship.

Here is an example where we create a \texttt{Point2D} class to define
point-specific methods, and then re-implement our Circle class to
\textbf{have} a \texttt{Point2D} to represent its center.

    \begin{Verbatim}[commandchars=\\\{\}]
{\color{incolor}In [{\color{incolor} }]:} \PY{c+cp}{\PYZsh{}}\PY{c+cp}{include} \PY{c+cpf}{\PYZlt{}iostream\PYZgt{}}
\end{Verbatim}


    \begin{Verbatim}[commandchars=\\\{\}]
{\color{incolor}In [{\color{incolor} }]:} \PY{k}{class} \PY{n+nc}{Point2D}
        \PY{p}{\PYZob{}}
        \PY{k}{private}\PY{o}{:}
            \PY{k+kt}{double} \PY{n}{x\PYZus{}}\PY{p}{;}
            \PY{k+kt}{double} \PY{n}{y\PYZus{}}\PY{p}{;}
         
        \PY{k}{public}\PY{o}{:}
            \PY{c+c1}{// A default constructor}
            \PY{n}{Point2D}\PY{p}{(}\PY{p}{)}\PY{p}{\PYZob{}}\PY{p}{\PYZcb{}}\PY{p}{;}
         
            \PY{n}{Point2D}\PY{p}{(}\PY{k+kt}{double} \PY{n}{x}\PY{p}{,} \PY{k+kt}{double} \PY{n}{y}\PY{p}{)}\PY{o}{:} \PY{n}{x\PYZus{}}\PY{p}{(}\PY{n}{x}\PY{p}{)}\PY{p}{,} \PY{n}{y\PYZus{}}\PY{p}{(}\PY{n}{y}\PY{p}{)}\PY{p}{\PYZob{}}\PY{p}{\PYZcb{}}\PY{p}{;}
         
            \PY{c+c1}{// An overloaded output operator}
            \PY{k}{friend} \PY{n}{std}\PY{o}{:}\PY{o}{:}\PY{n}{ostream}\PY{o}{\PYZam{}} \PY{k}{operator}\PY{o}{\PYZlt{}}\PY{o}{\PYZlt{}}\PY{p}{(}\PY{n}{std}\PY{o}{:}\PY{o}{:}\PY{n}{ostream}\PY{o}{\PYZam{}} \PY{n}{out}\PY{p}{,} \PY{k}{const} \PY{n}{Point2D} \PY{o}{\PYZam{}}\PY{n}{point}\PY{p}{)}
            \PY{p}{\PYZob{}}
                \PY{n}{out} \PY{o}{\PYZlt{}}\PY{o}{\PYZlt{}} \PY{l+s}{\PYZdq{}}\PY{l+s}{(}\PY{l+s}{\PYZdq{}} \PY{o}{\PYZlt{}}\PY{o}{\PYZlt{}} \PY{n}{point}\PY{p}{.}\PY{n}{x\PYZus{}} \PY{o}{\PYZlt{}}\PY{o}{\PYZlt{}} \PY{l+s}{\PYZdq{}}\PY{l+s}{, }\PY{l+s}{\PYZdq{}} \PY{o}{\PYZlt{}}\PY{o}{\PYZlt{}} \PY{n}{point}\PY{p}{.}\PY{n}{y\PYZus{}} \PY{o}{\PYZlt{}}\PY{o}{\PYZlt{}} \PY{l+s}{\PYZdq{}}\PY{l+s}{)}\PY{l+s}{\PYZdq{}}\PY{p}{;}
                \PY{k}{return} \PY{n}{out}\PY{p}{;}
            \PY{p}{\PYZcb{}}
         
        \PY{p}{\PYZcb{}}\PY{p}{;}
\end{Verbatim}


    \begin{Verbatim}[commandchars=\\\{\}]
{\color{incolor}In [{\color{incolor} }]:} \PY{n}{Point2D} \PY{n}{p} \PY{o}{=} \PY{n}{Point2D}\PY{p}{(}\PY{l+m+mi}{4}\PY{p}{,}\PY{l+m+mi}{5}\PY{p}{)}
\end{Verbatim}


    \begin{Verbatim}[commandchars=\\\{\}]
{\color{incolor}In [{\color{incolor} }]:} \PY{n}{std}\PY{o}{:}\PY{o}{:}\PY{n}{cout} \PY{o}{\PYZlt{}}\PY{o}{\PYZlt{}} \PY{n}{p} \PY{o}{\PYZlt{}}\PY{o}{\PYZlt{}} \PY{n}{std}\PY{o}{:}\PY{o}{:}\PY{n}{endl}\PY{p}{;}
\end{Verbatim}


    \begin{Verbatim}[commandchars=\\\{\}]
{\color{incolor}In [{\color{incolor} }]:} \PY{k}{class} \PY{n+nc}{Circle2}\PY{o}{:} \PY{k}{public} \PY{n}{Shape} \PY{p}{\PYZob{}}
            \PY{n}{Point2D} \PY{n}{center\PYZus{}}\PY{p}{;}
            \PY{k+kt}{double} \PY{n}{radius\PYZus{}}\PY{p}{;}
            
          \PY{k}{public}\PY{o}{:}
            \PY{n}{Circle2}\PY{p}{(}\PY{k+kt}{double} \PY{n}{x}\PY{p}{,} \PY{k+kt}{double} \PY{n}{y}\PY{p}{,} \PY{k+kt}{double} \PY{n}{radius}\PY{p}{)}\PY{o}{:}\PY{n}{center\PYZus{}}\PY{p}{(}\PY{n}{x}\PY{p}{,} \PY{n}{y}\PY{p}{)}\PY{p}{,} \PY{n}{radius\PYZus{}}\PY{p}{(}\PY{n}{radius}\PY{p}{)}\PY{p}{\PYZob{}}\PY{p}{\PYZcb{}}\PY{p}{;}
            
            \PY{c+c1}{// A default constructor}
            \PY{n}{Circle2}\PY{p}{(}\PY{p}{)}\PY{p}{\PYZob{}}\PY{p}{\PYZcb{}}\PY{p}{;}
            
            \PY{k+kt}{double} \PY{n+nf}{GetArea}\PY{p}{(}\PY{p}{)}\PY{p}{\PYZob{}}
                \PY{k}{return} \PY{n}{M\PYZus{}PI} \PY{o}{*} \PY{n}{radius\PYZus{}} \PY{o}{*} \PY{n}{radius\PYZus{}}\PY{p}{;}
            \PY{p}{\PYZcb{}}\PY{p}{;}
            
            \PY{k}{const} \PY{n}{Point2D}\PY{o}{*} \PY{n+nf}{GetLocation}\PY{p}{(}\PY{p}{)}\PY{p}{\PYZob{}}
                \PY{k}{return} \PY{o}{\PYZam{}}\PY{n}{center\PYZus{}}\PY{p}{;}
            \PY{p}{\PYZcb{}}
            
        \PY{p}{\PYZcb{}}
\end{Verbatim}


    \begin{Verbatim}[commandchars=\\\{\}]
{\color{incolor}In [{\color{incolor} }]:} \PY{n}{Circle2} \PY{n}{circle} \PY{o}{=} \PY{n}{Circle2}\PY{p}{(}\PY{l+m+mi}{4}\PY{p}{,}\PY{l+m+mi}{3}\PY{p}{,}\PY{l+m+mi}{2}\PY{p}{)}\PY{p}{;}
\end{Verbatim}


    \begin{Verbatim}[commandchars=\\\{\}]
{\color{incolor}In [{\color{incolor} }]:} \PY{n}{circle}\PY{p}{.}\PY{n}{GetArea}\PY{p}{(}\PY{p}{)}
\end{Verbatim}


    \begin{Verbatim}[commandchars=\\\{\}]
{\color{incolor}In [{\color{incolor} }]:} \PY{n}{std}\PY{o}{:}\PY{o}{:}\PY{n}{cout} \PY{o}{\PYZlt{}}\PY{o}{\PYZlt{}} \PY{o}{*}\PY{p}{(}\PY{n}{circle}\PY{p}{.}\PY{n}{GetLocation}\PY{p}{(}\PY{p}{)}\PY{p}{)} \PY{o}{\PYZlt{}}\PY{o}{\PYZlt{}} \PY{n}{std}\PY{o}{:}\PY{o}{:}\PY{n}{endl}\PY{p}{;}
\end{Verbatim}


    \begin{Verbatim}[commandchars=\\\{\}]
{\color{incolor}In [{\color{incolor} }]:} \PY{n}{Circle2} \PY{n}{circle2} \PY{o}{=} \PY{n}{Circle2}\PY{p}{(}\PY{p}{)}\PY{p}{;}
\end{Verbatim}


    \begin{Verbatim}[commandchars=\\\{\}]
{\color{incolor}In [{\color{incolor} }]:} \PY{c+c1}{// What value will this return?}
        \PY{n}{std}\PY{o}{:}\PY{o}{:}\PY{n}{cout} \PY{o}{\PYZlt{}}\PY{o}{\PYZlt{}} \PY{o}{*}\PY{p}{(}\PY{n}{circle2}\PY{p}{.}\PY{n}{GetLocation}\PY{p}{(}\PY{p}{)}\PY{p}{)} \PY{o}{\PYZlt{}}\PY{o}{\PYZlt{}} \PY{n}{std}\PY{o}{:}\PY{o}{:}\PY{n}{endl}\PY{p}{;}
\end{Verbatim}



    % Add a bibliography block to the postdoc
    
    
    
    \end{document}
